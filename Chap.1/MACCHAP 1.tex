%!TEX TS-program = xelatex
%!TEX encoding = UTF-8 Unicode
%times,

\documentclass[aps,showpacs,twocolumn]{revtex4}%
\usepackage{amsmath}
\usepackage{amsfonts}
\usepackage{amssymb}
\usepackage{graphicx}
\usepackage{amsthm}
\usepackage[colorlinks,linkcolor=blue,anchorcolor=blue,citecolor=blue,urlcolor=red]%
{hyperref}
\usepackage{pstricks}
\usepackage{pst-plot}
\usepackage{floatflt}
\usepackage{mathrsfs}
\usepackage{dcolumn}
\usepackage{bm}
\usepackage{epsfig}%
\usepackage{tikz}
\usepackage[version=3]{mhchem}
\usepackage{xeCJK}
\usepackage{fancyhdr}
\newtheorem{defi}{Definition}
\setCJKmainfont[BoldFont={STHeiti},ItalicFont=STKaiti]{STSong}
\setcounter{MaxMatrixCols}{30}
\providecommand{\U}[1]{\protect\rule{.1in}{.1in}}
\pagestyle{fancy}
\chead{悖论研究:期末论文\\ \bf 纠缠态与定域实在论:真的矛盾了吗}
\rhead{\today}
\lhead{\textsc{Laser Young}}
\begin{document}
\title{纠缠态与定域实在论:真的矛盾了吗\\ ——Einstein与Quantum Theory的战争}
\author{Laser Young$^{1}$\\ 1300011317}
\affiliation{$^{1}$School of Physics, Peking University, Beijing 100871, P. R. China}
\date{\today}
\begin{abstract}

Einstein反对量子理论这并不是一个新闻,然而他反对的理由却很少有人能完整地讲清楚。事实上,Einstein的论证基于定域物理实在这么一件事上\cite{PhysRev.47.777}。其中,有两个核心思想:物理实在要素和相对论定域因果性。本文从历史上最早的EPR佯谬开始出发,用尽量贴近哲学思辨而不是理论推导的形式,仔细分析量子力学和定域实在论的世纪之争。

\end{abstract}

\pacs{03.65.Bz, 42.50.Dv}
\maketitle
\tableofcontents

\section{Introduction}

Einstein等人的理论确实很有说服力——一个经典的情况的很自然地推广。尤其是,定域因果性基本就是现代物理里面一个最为基本的思考点,甚至是我们广大物理学家的“信仰”\footnote{严格来说并不是信仰,而是一个基本假设,甚至可以说是最基本的假设,和欧几里德的公理一样基本}。然而如果同时承认定域因果律和物理实在论的话,就会与基本的量子力学思想产生悖论。又由于无论是量子力学还是Einstein的想法(狭义相对论范畴),都有各自发挥效应的领域,这两者必然部分的格子描述了物理世界。如果是没有矛盾的两个适用于不同尺度的物理理论,我们可以很欣慰地说我们对物理世界有了不同尺度的正确理解,剩下的就是找一个合适的描述方式将两者融洽的拼接在一起,就像当年我们对电磁理论做的那样:Maxwell电磁场方程把分别独立成立的电场、磁场理论给组合在一起了。然而,如果两个理论内部有不自洽的地方,那么无论如何也不可能把它们合为一个理论。就连Maxwell当年,也是解决了磁场理论中的一个小瑕疵之后才能做一个统一的表述的。

\ 

我们知道,量子力学在我们的生活中远比相对论更容易被感受到:如今生活中的计算机等等一系列电子设备的基石——晶体管,就是利用量子力学的观点设计并计算出来的。而生活中许多看似不合理的现象,比如人为什么不会顺着地板上分子的间隙掉下去等等问题,也可以用量子力学解释。如此“功能强大”的量子力学让人非常不希望怀疑他是错的,所以人们的想法是可能Einstein对量子力学的两点质疑可能并没有那么强的根据。然而,我们说了,因果律是“信仰”,那么唯一有可能不相容的地方就是物理实在论了。

\ 

\begin{defi}[物理实在论]
物理实在论的描述是这样的:任一可观测的物理量,作为物理实在的一个要素,它必定在客观上以确定的方式存在着。于是,一个完备的物理理论应当满足:一个系统在没有受到扰动时,可观测的物理量在客观上应当具有确定的数值。
\end{defi}

\ 

Einstein提出了EPR佯谬,考虑的是一对彼此满足某个守恒量的体系,在空间中相反的传播。距离足够远且在某一参考系时间足够近的时候,两者应该是因果无关的。如果测量得到其中一个的某个守恒量,则说明另一个的该守恒量的值是被确定的了。比如说,如果两者满足角动量守恒,我可以通过对某一个部分的角动量任意一个方向测量,证明另一者的该方向分量是一个物理实在要素,客观上确定的存在着。然而,量子力学要求,一个微观体系角动量的分量,在很多情况(比如该粒子自旋为$1/2$,如电子)下,彼此不能同时确定。这样的话,我们就知道:

\begin{enumerate}
\item 要么量子力学中的描述方式不完备,我们实际上可以同时知道这些量子力学里认为不可同时知道的参数
\item 要么,体系的两个部分即使处于类空的间隔,即因果律无关的情况,实际也可以发生其物理量的变化影响
\end{enumerate}

可以看出,这个第二条是完全与Einstein的想法相悖的。因此,Einstein说,量子力学只是一个\emph{统计学的描述理论},并不能完备的描述一个体系的态。更进一步,量子理论可以分为如下两个方面:

\begin{itemize}
\item Unitary Evolution Prodecure, 就像牛顿力学一样的动力学演化,可逆,是决定论的,与Einstein想法一致
\item Measurement, 量子测量,不可逆的,是概率的
\end{itemize}

量子力学必然有$R$过程发生,从而是反对决定论描述的。

\ 

而Einstein则从经典物理学事实来出发:随机性之所以存在,要么是由于自由度太多人们难于处理,要么是由于人们对其不感兴趣,有意的将其粗化从而简化处理,进行统计概率计算,但是归根结底它们是必然的,并不存在真正的偶然现象。

\ 

那么,谁错了呢?

\ 

谁错了这一点其实并没有定论,但至少,爱因斯坦物理实在论的表述存在一定问题,恰当的修正之后可以在很大程度上与量子力学保持融洽。历史上最为接近成功的是隐变量理论,而且直到如今,虽然最初的隐变量理论已经被Bell不等式干的体无完肤,但是一种新的基于拓扑的全局隐变量仍然有着渺茫的生机(尽管由于涉及太多的微分几何和拓扑学知识,我们这里不去详尽的说明,其物理图像我还是会简要提及)。



\section{隐变量与Bell不等式}

基于隐变量和定域物理实在论的理论是一个Einstein的简单想法。然而,可以看到的是,这类理论存在一个共性,而且这个共性是可以进行实验观测的。

\ 

假定有某个隐变量理论,某个测量的结果是由这个隐变量唯一决定的,只不过这个隐变量我们并不知道,而且它存在快速的变化,导致宏观上表现出了随机性。1964年的Bell,基于这点提出了一个不等式来研究这个问题。

\ 

比如说,我们现在的某个态,标记为$|\uparrow\rangle$,实际上是由一个多余的变量恒定的(实际上这个变量假定是一个数值,这是一个并不太完全的假定。但是,多个变量,亦或是算符表述,在这里影响很大)。由于我们了解实数的性质,可以对这个参数做一定处理,比如给定区间范围之类的。基于简单的代数表示,和“两个粒子存在守恒量”这个想法,可以构造自旋体系,满足

\[|P(a,b)-P(a,c)|\le 1+P(b,c) \]

如果存在隐变量的话。这个是可以通过实验统计平均数据的关系来仔细衡量的,实验也确实做出来了,如\cite{PhysRevLett.49.91}。

\ 

明显的,这个不等式被各种破坏。然而Bell不等式的定义存在问题,比如他只针对某一个参数,而不是某一个全局参数。

\ 

局域与全局的问题同样出现在凝聚态问题中。

\begin{quote}
However, the major limitation of Landau’s theory of phase transitions is that it is related to a local order parameter. In the past decade it has become clear that a series of phases of matter with so-called topological order do not have a local order parameter. For some (most) of them, a (highly) nonlocal order parameter can be defined, but it is unclear how a Landau-like theory of this order parameter can be developed. 

\quad -- B. ANDREI BERNEVIG with Taylor L. Hughes, \emph{TOPOLOGICAL INSULATORS AND TOPOLOGICAL SUPERCONDUCTORS}\cite{bernevig2013topological}
\end{quote}

\begin{widetext}
\begin{center}
\includegraphics[width = 12cm]{1.png}

图1:拓扑参数与非拓扑参数的区别。XYZ的曲率关联函数$\Gamma^{\alpha}_{\beta\gamma}$的局域解析性质可能存在不同,\cite{PhysRevE.86.046106}(9)有对此的详细计算。可以参考一份补充材料的计算过程,\cite{YY11,YY12,YY13}。
\end{center}
\end{widetext}

一个非定域的拓扑不变量可以带来非常酷炫的拓扑绝缘体和拓扑超导体,同样也可以带来非常酷炫的隐变量描述方式。基于拓扑的隐变量理论就此展开,相关综述见\cite{PhysRevE.86.046106}。

\section{量子纠缠?}

前面说的再多,实际上也只是一个局部系统的“叠加态”问题与物理实在论的矛盾而已。事实上,量子纠缠也会带来新的对Einstein想法的争执。

\ 

一个宏观世界的比喻是这么说的\cite{ZH}:

\begin{quote} 


富翁财产继承案

\ 

财产是一个可以观测的“物理实在”,它可以精确确定。有一个富翁有1亿元财产要被两个儿子继承,而且他已经明确的合法的声明这1亿元只会给两个儿子不会给其他人。然而,更进一步的由谁来继承多少,并没有确定,从而两个儿子各自的财产状况是不确定的。如果有一天,司法部门要彻查其中某一个儿子的资产,那么必须把这个继承的数值给出一个明确的估计,从而这个儿子的财产被测量所决定,取决于司法部门是怎么想的,是随机的。然而这样的话又会使另一个儿子的资产随之确定。总之,这跟人一种“客观实在不等价于客观单值确定”的感觉。
\end{quote}

换句话说呢,这有一种量子纠缠造成的不确定性的感觉。

\ 

另一方面,我们考虑一个类空间隔,如果多体量子有某种程度上的空间定域性,则类空间隔下各自发生的任何事件都满足统计独立,从而确保了一种乘积分解的形式他是一种独立性的表现,又是在类空间隔下定域的问题,所以说为类空间隔下的定语独立性。

\ 

按照量子信息的角度,量子纠缠满足两体系统(这是经典和量子里面精确可解的系统的最基本情况)类空间隔下关联测量的空间定域性欲量子态可分解为独立量子态Direct Product的形式,并且可以推广到多体情况,所以纠缠的本质是关联中的量子信息。上一节讲了隐变量,那么这里,两体系统存在纠缠的充要条件是两离子见的相位差是恒定的,不能容忍隐变量的存在。

\ 

总结如下:量子纠缠能够造成某一个可观测量在孤立情况下就是不确定的,但是没有食言明确否定隐变量的存在从而论证量子理论的完备性,尽管实验结果能够支持很多预测,也不能排除量子的或然性是否本质上不同于经典的或然性。所以,Einstein的关于物理实在论的表述存在问题:它不承认向感性带来的测量不确定性,不承认纠缠带来的不确定性,不承认测量方式($R$过程)带来不同的测量结果,认为物理体系是定域的;但是这并不是说物理实在论是不对的:比如说,全局的隐变量(如拓扑隐变量,见前文),就是可以涵盖量子理论的。

\section{Bell非定域性关系与纠缠}

可以定义一类叫做Bell型非定域性的关联。这种非定域性是一种空间非定域性,其严格发展经历可以参考\cite{YYPRL}。总体来说,这是一种基于Bell不等式出发的理论研究,研究的并不简单是Hidden Variable的问题,而是研究有纠缠的体系会怎么样。前面说过,仅仅隐变量理论在没仔细去讨论纠缠的度量的时候就已经产生了严格的区分,那么在引入纠缠态之后,空间非定域就可以利用Bell不等式来进行估计和衡量。

\ 

考虑多体粒子系统,将它们构造一组最大纠缠态,进行类空间隔下的测量(即无因果关系的测量)。很容易的就可以构造不同的不等式来进行这种非定域的现象的验证。这些不等式的狗仔都很具体,但是肯定是有局限性的。Bell非定域性和量子纠缠也不等价,只能作为某种度量。

\section{因果律问题}

这部分内容不得不涉及很多细节计算,我们这里就只针对结论进行说明。结论的得出可以参考\cite{RevModPhys.51.863},物理概要可以参考\cite{ZH}的第25讲。

\ 

量子力学是雨因果律相容的,这点毋庸置疑;但是,如果一定程度上,量子理论可以在量子涨落过程中,或者量子纠缠态的测量塌缩域关联塌缩的过程中,域相对论行定域因果律存在一定出入,这个出入还没有定论。按照量子理论的理解,实质性的因果关系存在于量子测量过程——$R$过程,但这个过程是与决定论体系存在矛盾的。实验上还没有办法区分这种或然性时表现出来的(就像混沌理论那样,一个由决定论的系统表现出高强度的各向同性和随机性,其随机检验程度达到了令人发指的程度;解析数论对此有详细解释,见\cite{JXSL})。最重要的是,这种矛盾很可能会带来基于相对论性的引力理论基本不可用量子力学的观点重整化。

\ 

因此,很多人试图建立一种广义量子理论,从而定义一个固定的时空几何理论,通过引入曲率和度规完善量子理论的建立,量子动力学变数从而就可以约化到量子涨落。从而,两个基于集合描述的理论就可以存在相容的可能,彼此之间可以建立所谓的Ads/CFT对偶。

\section{小结}

20世纪初的两朵乌云——相对论和量子力学,各自在各自的领域给出了美妙而卓越的贡献。相对论的确立让我们能够使用GPS定位,让我们能了解到存在黑洞这样可怕的星体,让我们知道Kerr黑洞与白洞、Einstein-Rosen桥等等一系列问题,为我们在大尺度上能够了解我们生存的环境;而量子理论让我们领会到诸多神奇的材料,超导体,半导体,都用量子框架来描述从而服务人类,并且在极高能的情况下,标准模型给了我们深入理解微观世界的可能。然而,两个如此曼妙的理论却如此的格格不入,不相容,甚至频频引发悖论,着实令人着迷的想要深入理解其中问题,提出解决方案。

\ 

然而,几代物理大师努力之下,这两个问题仍然存在一定的排斥,而且最近几年来对这里悖论的研究越来越少。毕竟,如果不是去探索一些灰色区域,大家只要掌握好一项内容就足矣做出成果。

\ 

本文的目的就是通过对这些悖论的一个浅层概括,试图引导一种简单的量子-相对论观念,激发大家重新研究这种悖论的兴趣,并提供一些学界主流的看法,和一些个人见解。

\section{结语} 

作为一篇通选课——《悖论研究》的期末论文,我觉得这个选题可能有失偏颇:稍微的过于抽象,而且和我们课上学习的悖论知识关联性不高。但是,作为自然科学的前沿领域,如果能和哲学思辨的领域联系起来,我想这也许也是悖论研究课的一个初衷吧。感谢陈波老师。

\ 

同时,在我完成这篇期末论文的时候,我对这个问题的认识也在不断变化,所以如果有哪些部分出现了略显矛盾的说法,见谅,我已经尽可能的修正了这里面的问题了。

\begin{acknowledgments}
本文参考了诸多资料,感谢APS(American Physics Society)对论文的刊载,感谢北京大学对APS文献的购买,感谢陈波老师加深了我对悖论的认识\footnote{希望老师上课的时候提问能问点有难度的东西,不然问很基础的问题让我感觉有一种被瞧不起的感觉。诸如“红帽子”问题等等小学奥数问题我觉得课上所有的人都应该会做吧。},感谢余佳晨同学关于量子讨论班中与本题相关的部分与我的讨论交流,感谢张永德老师写的并不怎么样的量子信息书。
\end{acknowledgments}

%\bibliographystyle{plain}
\bibliography{reference-1}


\end{document}

